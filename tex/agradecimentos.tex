\newcommand{\forceindent}{\leavevmode{\parindent=1em\indent}}


Os autores gostariam de começar agradecendo em conjunto a todos aqueles que nos ajudaram, sendo na confecção desse trabalhou, ou sendo no nosso dia-a-dia. Ao Prof. Dr. Geraldo Pereira por acreditar no potencial do nosso projeto e se dispor a nos ajudar mesmo fora de sua área de pesquisa. A OpenAI por fornecer bibliotecas de ponta para o desenvolvimento de um futuro com inteligência artificial geral que beneficiará toda a humanidade. A todas a rádios de Jazz e Lo-fi do Youtube, com suas músicas tranquilas que nos forneceram paciência para conseguirmos fazer os modelos funcionarem. Aos nossos amigos que se fazem presente para nos apoiar diariamente e levaremos para a vida: André Marques, Augusto Brandão, Bernardo Carsten, Camila Sidersky, Claudio Segala, Daniel Bemerguy, Fernando Sobral, Johannes Peter, Léo Akira, Marcelo Araújo, Mateus Bittencourt, Pedro Saman, Ricardo Nunes e Rodrigo Navarro.

\vspace{5mm}

O autor Khalil Carsten agradece com a seguinte mensagem: 

\vspace{5mm}

\begin{quote}

\forceindent ``Direciono, prioritariamente, ao meu colega de curso e amigo, Renato Avellar Nobre, meus agradecimentos pela sua incrível lógica cientifica e capacidade de escrita que me conduziram em momentos de confusão, além de sua determinação e disciplina nas quais me mantiveram focado e disposto durante todo o desenvolvimento deste trabalho. 

\forceindent Com muito carinho também agradeço aos meus pais Benivaldo do Nascimento Junior e Germana Magalhães Carsten por todo conforto emocional e financeiro que me proveram para que esse trabalho fosse concluído, sem eles nada disso teria sido possível. 

\forceindent Também aos meus irmãos Caio Lemos e Carmel Carsten por serem meus parceiros de vida e me proporcionarem momentos de alegria e divertimento. 

\forceindent Da mesma forma agradeço aos meus Tios Carla Miranda e Alexandre Miranda por sempre me receberem como um filho, juntamente com meus primos Gabriel Miranda e Bernardo Miranda por se disporem a me ouvir em tranquilos cafés da manhã em sua casa.''

\end{quote}

\newpage

O autor Renato Nobre agradece com a seguinte mensagem: 

\vspace{5mm}

\begin{quote}

\forceindent ``Gostaria de começar agradecendo ao Khalil Carsten por toda a amizade e parceria desenvolvida durante esses 5 anos de graduação, pela paciência para me aguentar, pelos códigos incríveis, e por aceitar o desenvolvimento deste trabalho megalomaníaco.

\forceindent A meus pais Ademar Thadeu Murta Nobre e Martha Maria Nobre Avellar, e meus avós Nair Renata Nobre Avellar e Américo José Avellar, pela confiança, apoio, orientação, e suporte para a realização desse curso e deste trabalho, nada do que eu sou hoje seria possível sem a contribuição de vocês. O mesmo se estende também para toda a minha família, meus irmãos e tios, que também sempre estiveram ao meu lado e criaram essa família acolhedora e presente a qual eu pertenço. 

\forceindent Gostaria de agradecer também, com todo meu coração, a Giovanna Mundstock, você é o amor da minha vida. Obrigado por todos os momentos de apoio, compreensão e incentivos. Cada dia ao seu lado eu cresço mais um pouco e você me motiva a ir mais longe.

\forceindent Também não podem faltar agradecimentos ao meu grande amigo Mario Luiz Menel da Cunha, por tirar um tempo para ler este trabalho e nos retornar com diversas dicas e elogios. Nem aos meus amigos de São Paulo, pelos conselhos de investimento e pelas conversas de bar que tivemos em 2017 que me motivaram na criação desse projeto: Daniel Miranda, Rodrigo Nogueira e Guilherme Carvalho. 

\forceindent Finalmente, agradeço ao Prof. Dr. Guilherme Novaes Ramos por fornecer uma oportunidade de pesquisa para mim no meu segundo semestre de graduação, e me orientar em diversos outros momentos ao longo desses 5 anos. Essa oportunidade me apresentou a aprendizagem de máquina e o aprendizado por reforço, o ramo no qual viria a dedicar todo o resto da minha graduação.''
\end{quote}

\vspace{90mm}