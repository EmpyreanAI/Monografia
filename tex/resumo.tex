
Os mercados de ações desempenham um papel crucial na economia, permitindo o crescimento de empresas e possibilitando uma geração de rendimento para seus investidores. Na literatura de aprendizado de máquina para mercados financeiros, múltiplas ferramentas e técnicas foram propostas e aplicadas para analisar o comportamento geral do mercado. No entanto, entender as regras intrínsecas do funcionamento da bolsa de valores, com a possibilidade de gerar lucros, está longe de ser uma tarefa trivial. Abordando esse desafio, este trabalho propõe o Hare: um serviço de investimento com técnicas híbridas, orientado a agentes racionais autônomos, para negociar ativos no mercado de ações. O Hare oferece um serviço confiável com alta precisão e estabilidade no processo de tomadas de decisão, se baseando em análises técnicas e fundamentais. O cerne de funcionamento do Hare é sua modelagem utilizando um agente racional capaz de perceber o mercado e agir de forma autônoma com base em suas decisões. Para tal dois módulos principais foram implementados com o objetivo de fornecer uma racionalidade ao agente: (i) o \acrfull{MPM}, responsável por prever a movimentação de um ativo; e (ii) o \acrfull{MAR}, responsável por utilizar as informações de predição ao seu favor, distribuindo seus recursos entre os ativos disponíveis para tentar gerar o maior lucro possível com o menor risco. Como prova de conceito, o Hare foi projetado para operar gerenciando um portfólio no mercado de ações da B3. Os resultados avaliados do \acrshort{MPM} demonstraram que o serviço 
é capaz de prever o ganho ou perda de valor no preço de uma ação, quando comparado com sua média da janela de tempo analisada, com uma acurácia de $82\%$ no pior caso e $94\%$ na melhor situação. Ademais, o \acrshort{MAR} foi capaz de obter uma rentabilidade de $11,74\%$ gerenciando um portfólio com $3$ ativos no período de tempo analisado. Ainda, Hare foi capaz de superar a rentabilidade de investimentos de renda fixa e de portfólios construídos com base na Variância Média de Markowitz.%Isso demonstra que o Hare é uma opção para o gerenciamento de investimentos viável, sendo capaz de superar a rentabilidade de investimentos de renda fixa e de portfólios construídos com base na Variância Média de Markowitz.