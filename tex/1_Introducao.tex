Mercados financeiros assumem um papel essencial no comportamento da economia de um país \cite{investments, gomes1997bolsa, elton2012moderna}. Tais mercados tornam fácil para acionistas negociarem seus ativos financeiros, contribuindo para o crescimento das empresas e corporações, e gerando oportunidades para investidores possuírem rentabilidades sobre seus ativos \cite{investments}. O estudo desses mercados com o objetivo de realizar predições sobre seu movimento, pode ser capaz de aumentar o lucro e a rentabilidade dos investidores, além de prover um melhor entendimento das movimentações e valores de um determinado ativo em avaliação \cite{review}. Sendo assim, serviços de negociações baseados em modelos de decisão estão ganhando espaço no mercado financeiro nacional e internacional \cite{review}. 

Entre os vários tipos de mercados financeiros em que o investidor pode alocar os ativos, o mercado de ações se destaca em termos de popularidade \cite{investments}. Esse tipo de mercado negocia frações das empresas e corporações, denominadas ações\footnote{Quando um número específico de ações está sendo negociado elas são denominadas de posições, é dito que o investidor está comprando ou vendendo posições das ações de uma empresa. Por outro lado, quando um número indeterminado de ações está sendo negociado, a terminologia utilizada é de nomear essas ações de ativos \cite{investments}.}, ou diversos outros tipos de instrumentos financeiros, tais como câmbio, ouro e commodities \cite{elton2012moderna}. Os investidores que negociam ativos no mercado normalmente são guiados por alguma forma de predição gerada por uma análise oportunista do meio em que estão inseridos. Alguns tipos comuns de análises realizadas são análise: da situação do governo, da situação do país, da situação da empresa detentora dos ativos, ou até mesmo das variações de preço durante um intervalo de tempo \cite{review}.

Considerando o potencial que as técnicas de \acrfull{ML} podem oferecer, seu uso é um dos caminhos viáveis para prover serviços relacionados com a predição de um ativo no mercado financeiro. O processo de prever movimento dos mercados é uma área de pesquisa em crescimento dentro dos estudos de \acrshort{ML} \cite{review}. Esse crescimento pode oferecer possibilidades para todos que buscam obter lucro e rentabilidade com suas finanças.

Estudos recentes vêm realizando pesquisas para prever os mercados financeiros utilizando diversas abordagens e modelos de \acrshort{ML}~\cite{google_trends, cesarone2011portfolio, predicting_direction_svm,  hybrid_forecasting, gabased_svm, forecasting_returns, nn_forecasting}. Tais abordagens são divididas em duas categorias principais: (i) \textit{análise fundamental}, na qual o valor da companhia que é responsável por definir o preço da ação, e não a ação em si; e (ii) \textit{análise técnica} na qual a predição do preço futuro de uma ação é realizada se baseando no estudo de seus preços e indicadores passados e presentes~\cite{fundamental_technical_analysis}.

Nos trabalhos analisados, percebe-se que há estudos que são baseados em técnicas de \textit{análise de sentimento} com o objetivo de prever o valor de um ativo \cite{google_trends, review}. Outros trabalhos utilizam o comportamento de variáveis macroeconômicas\footnote{Medidas que indicam as variáveis agregadas de todo o país.} ou índices de mercados ao redor do mundo \cite{random_forest_macroeconomic, forecasting_bayesian, review}. Entretanto, em grande maioria os trabalhos utilizam dados técnicos dos ativos em formato de séries temporais para criar modelos ou indicadores técnicos \cite{fusionportifolio, cesarone2011portfolio, airms, ga_optimized_lstm}. Uma grande parcela dos trabalhos encontrados na literatura não utiliza predições em cima de ações de empresas e corporações disponíveis no mercado, preferindo outros tipos de ativos financeiros, tais como cripto-moedas \cite{cryptocurrency}, moedas estrangeiras \cite{classification_dnn, airms}, índices de mercados \cite{forecasting_bayesian, increase_decrease, ga_optimized_lstm} e commodities \cite{classification_dnn}. Além disso, quando um estudo é de fato focado no mercado de ações é, usualmente, baseado nos mercados Asiáticos e Europeus \cite{review}. Não obstante, todos os trabalhos mencionados não exploram serviços híbridos, baseados em análise técnica em conjunto com a análise fundamental, além de não prover um mecanismo de agir em cima das decisões.

Enquanto a maioria das pessoas na área acreditam que os mercados de alguma maneira são previsíveis, também existem opiniões contrárias \cite{review}. Alguns pesquisadores acreditam na hipótese do mercado eficiente, que afirma que o preço dos ativos é um reflexo de toda informação acessível daquele ativo naquele instante de tempo. Dessa forma, as informações do presente momento não alterariam o futuro de um ativo e, por consequência, a previsão de retornos futuros não seria possível \cite{fama}. Portanto, para o desenvolvimento deste trabalho, não será considerado tal hipótese, visto que se ela fosse verdade, não faria sentido realizá-lo. Sendo assim, este trabalho se sustenta na hipótese de que é possível criar um serviço com alta precisão nas predições e estabilidade para negociar ativos no mercado de ações com base nas análises técnicas e fundamentais das empresas.

\section{Objetivos}

Este trabalho propõe o Hare\footnote{O nome escolhido significa Lebre em inglês. O motivo de escolha desse animal para nomear o sistema é dado pelo senso comum de que a Lebre é um animal rápido e ágil em suas movimentações e decisões, características que o serviço proposto busca promover.}, um serviço híbrido, autônomo, orientado a agente para negociar ativos no mercado financeiro desejado. O Hare é baseado em agentes racionais, entidades computacionais capazes de observar o mercado e agir de forma autônoma em cima de suas percepções. Racionalidade para o agente, é definida neste escopo, como a capacidade do agente escolher as ações que maximizarão seu lucro e diminuirão seus riscos. Para atingir esse objetivo, o Hare foi projetado em uma estrutura modular que contém: (i) um módulo preditor utilizando uma unidade de redes neurais recorrentes baseada em portões chamada \acrfull{LSTM}, juntamente com uma biblioteca de hiper-parametrização denominada Hyperopt; (ii) um módulo de controle de risco utilizando informações de busca do \emph{Google Trends}\footnote{https://trends.google.com}, informações de notícias e relatórios financeiros; e (iii) um módulo atuador racional treinado utilizando um algoritmo de aprendizado por reforço denominado \acrlong{DDPG}, que tem como finalidade gerenciar os recursos disponíveis ao agente. Portanto, o Hare provê um serviço de ponta-a-ponta para investimento em ações na bolsa de valores; da análise dos dados até as ordens de negociação na corretora.

\section{Contribuição}

As principais contribuições deste trabalho, em comparação com outros trabalhos na literatura, são destacadas a seguir:

\begin{enumerate}
    \item O modelo dos agentes possui módulos preditores especializados. Tais preditores são treinados em uma ação específica do conjunto de ações escolhido pelo usuário em seu portfólio.
    \item As análises de movimento do ativo são híbridas por meio da análise fundamental para administrar os riscos de um ativo, bem como da análise técnica que utiliza a série histórica para prever a direção do ativo.
    \item O serviço aloca recursos no mercado. Neste caso, o serviço proposto não para na predição, sendo também responsável por gerenciar recursos com conhecimento aprendido por métodos de aprendizagem por reforço.
\end{enumerate}

\section{Estrutura do Trabalho}

A estrutura deste trabalho está organizada do seguinte modo. O Capítulo \ref{cap:fund} estabelece a base teórica necessária para a compreensão do desenvolvimento do trabalho, apresentando os conceitos básicos de economia e mercado, e conceitos da área de aprendizado de máquina inerentes a aplicação do serviço. O Capítulo \ref{cap:rwork} apresenta os trabalhos relacionados no campo de predição de mercados financeiros, que utilizam análises fundamentais e técnicas em suas abordagens. No Capítulo \ref{cap_dev} o Hare é apresentado, com foco em sua implementação como serviço de investimento autônomo, racional e baseado em agentes. Posteriormente, no Capítulo \ref{cap_res} é apresentada a metodologia para experimentação dos módulos do Hare, responsáveis pela sua validação. Por fim, o Capítulo \ref{cap_conc} apresenta as conclusões e as oportunidades para trabalhos futuros.

